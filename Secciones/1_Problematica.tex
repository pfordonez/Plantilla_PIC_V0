\chapter{Problema de investigación}
\label{Problemática}

%\section{Situación Problemática}
%\begin{itemize}
%    \item Primer párrafo en donde se sitúa la definición de las variable(s)
%    \item Segundo párrafo que contenga un abstracto bibliométrico sobre la variable (estado del arte RSL) (Ej. \textbf{5-10 estudios seleccionados})
%    \item Contexto del primer estudio (causa/efecto..)
%    \item Contexto del segundo estudio (causa/efecto)
%    \item Contexto ...... [Número de estudios que hayan citado]
%    \item Penúltimo párrafo --> El efecto mas trascendental que impactaría el problema (variable de enfoque)
%    \item Finalmente, una solución generalizada (de la variable)
%\end{itemize}


\section{Pregunta de Investigación}

%\textbf{La interrogante:} ¿? es la pregunta clave que se planteará.

%\textbf{Variable(1) o variables:} la variable o variables que forman parte del estudio. En el caso de un estudio descriptivo será una variable, mientras que en un estudio correlacional serán dos variables.

%\textbf{Enlace o relacionante:} el vínculo con el cual se relaciona las variables.

%\textbf{Población:} es generalmente la colección de individuos u objetos que son el foco principal de la investigación científica, y que serán observados, encuestados o medidos.

%\textbf{Delimitación espacial:} el lugar o zona geográfica que comprende el estudio. También comprende el ámbito específico de estudio, como por ejemplo puede ser una empresa determinada o conjunto de negocios (como los cinemas).

%\textbf{Delimitación temporal:} el período de tiempo que comprende el estudio.


%%%%%%%%%%%%%%%%Ejemplo%%%%%%%%%%%%


%"¿Cómo afecta la implementación de una plataforma de gestión de proyectos basada en la nube la colaboración y eficiencia de equipos distribuidos en comparación con las herramientas tradicionales de gestión de proyectos?" 

 %   Cuales son las PI derivadas ¿ ?
 %       \begin{itemize}
  %          \item Como se mide la afectacion (Test de impacto denomidao TA) var cuantitativa 
  %          \item Cual es la eficiencia (X,Z)
  %          \item Como establecer o medir la colaboracion (X, Y, Z)
  %          \item Cuales seria las herramientas tradicionales (Scrum, RUP)
  %      \end{itemize}


    %¿En que medida de porcentaje la el test de aceptación  afecta la implementación de una plataforma de gestión de proyectos basada en la nube la colaboración y eficiencia de equipos distribuidos en comparación con las herramientas tradicionales de gestión de proyectos como Kanban?

 \textbf{ ¿Que medida en porcentaje resulta el test de aceptación que  afecta la implementación de una plataforma de gestión de proyectos basada en la nube de colaboración Kanbam?} \\

Esta pregunta de investigación de justifica objetivamente por el  porcentaje del test de aceptación para la afección de la implementación. El impacto se observará mediante la aceptación del equipo con la herramienta Kanbam y finalmente los nuevos resultados en las tareas  se alineado con los objetivos del proyecto.


 