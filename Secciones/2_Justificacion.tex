\chapter{Justificación}
\label{Justificación}

%Proyecto: Desarrollo de una Plataforma de Gestión de Proyectos Basada en la Nube

\paragraph{Justificación Teórica}
La justificación teórica se refiere a la base conceptual y los principios cientficos que sustentan el proyecto. Incluye: 

\begin{itemize}
    \item \textbf{Revisión de la literatura}: Se revisan estudios y publicaciones sobre gestión de proyectos, incluyendo metodologías ágiles como  Kanban, así como teorías sobre la colaboración en equipos distribuidos.
    \item \textbf{Fundamentos teóricos}: Se fundamenta en teorías de gestión de proyectos, colaboración en línea, y computación en la nube. Se citan trabajos de autores como Harold Kerzner en gestión de proyectos y estudios recientes sobre la eficacia de herramientas de colaboración en la nube.
    \item \textbf{Estado del arte}: Análisis de plataformas actuales de gestión de proyectos como Jira, Asana y Trello, identificando sus fortalezas y debilidades.

\end{itemize}
\textbf{Ejemplo de contenido}: "Las teorías de gestión de proyectos de Harold Kerzner (2017) subrayan la importancia de la planificación y la colaboración efectiva. Estudios recientes (Smith et al., 2020) muestran que las herramientas de gestión basadas en la nube pueden aumentar la eficiencia en equipos distribuidos en un 25\%."

\paragraph{Justificación Práctica}
La justificación práctica se enfoca en la utilidad y la aplicabilidad del proyecto. Se trata de demostrar la relevancia y la necesidad del proyecto en un contexto real. Incluye: 

\begin{itemize}
    \item \textbf{Problemas o necesidades}: Identificación de la necesidad de una plataforma que permita a los equipos distribuidos gestionar proyectos de manera más efectiva, con características específicas como la integración de comunicación en tiempo real y herramientas avanzadas de reporting.
    \item \textbf{Beneficios}: La plataforma propuesta permitirá una mayor eficiencia en la gestión de tareas, mejor comunicación entre los miembros del equipo y un seguimiento más preciso del progreso del proyecto. Esto reducirá los costos operativos y mejorará la productividad.
    \item \textbf{Aplicaciones}: Utilización en empresas de tecnología, agencias de marketing, y cualquier organización que trabaje con equipos distribuidos o gestione múltiples proyectos simultáneamente.
    \item 


\end{itemize}
\textbf{Ejemplo de contenido}: "En una encuesta realizada por Project Management Institute (2019), el 60\% de las empresas informaron retrasos significativos en proyectos debido a la falta de herramientas de gestión adecuadas. Nuestra plataforma abordará esta necesidad proporcionando una solución integral que combina gestión de tareas, comunicación en tiempo real y reporting avanzado."

\paragraph{Justificación Metodológica}
La justificación metodológica describes los métodos y técnicas que se utilizaran para llevar cabo el proyecto. Incluye: 

\begin{itemize}
    \item \textbf{Métodos de investigación}: Uso de encuestas y entrevistas con potenciales usuarios para identificar características clave y requerimientos. Análisis de datos para entender las necesidades y preferencias de los usuarios. Metodos de aceptacion tecnológica etc..

    \item \textbf{Técnicas y herramientas}: Desarrollo ágil utilizando Scrum. Herramientas de desarrollo como React para el frontend, Node.js para el backend y servicios de AWS para la infraestructura en la nube.
    \item \textbf{Plan de trabajo}: El proyecto se dividirá en varias fases: investigación inicial (2 semanas), diseño y planificación (4 semanas), desarrollo (6 semanas), pruebas (2  semanas) y despliegue (2 semanas). Cada fase incluirá iteraciones y revisiones periódicas con los stakeholders.


\end{itemize}
\textbf{Ejemplo de contenido}: "Utilizaremos la metodología Scrum para asegurar un desarrollo iterativo e incremental, con sprints de 2 semanas y revisiones periódicas. Las herramientas elegidas, como React y Node.js, permitirán una construcción rápida y flexible del frontend y backend, mientras que AWS proporcionará una infraestructura escalable y segura."

Este enfoque integral asegura que el proyecto esté bien fundamentado teóricamente, responda a necesidades prácticas reales y esta metodológicamente preparado para su desarrollo y éxito.

 
