\chapter{Objetivos}
\label{Objetivos}

\subsubsection{Objetivo General}

\paragraph{¿Qué es?}

El objetivo general describe de manera amplia y comprensiva lo que se pretende alcanzar con el proyecto. Representa el propósito principal y la meta final del proyecto.

\paragraph{¿Cómo se elabora?}

\begin{enumerate}
    \item \textbf{Claridad}: Debe ser claro y comprensible. Evita el uso de términos ambiguos.
    \item \textbf{Amplitud}: Debe abarcar el alcance completo del proyecto.
    \item \textbf{Realismo}: Debe ser alcanzable y realista dentro del contexto y los recursos disponibles.
    \item \textbf{Redacción}: Generalmente se formula utilizando un verbo en infinitivo que indique una acción amplia, como "desarrollar", "mejorar", "crear", etc.
\end{enumerate}
\textbf{Ejemplo}: "Desarrollar una plataforma de gestión de proyectos basada en la nube para mejorar la colaboración y eficiencia en equipos distribuidos."

\subsubsection{Objetivos Específicos}

\paragraph{¿Qué son?}

Los objetivos específicos desglosan el objetivo general en metas más concretas y detalladas. Cada objetivo específico aborda un aspecto particular del proyecto y contribuye al logro del objetivo general.

\paragraph{¿Cómo se elaboran?}

\begin{enumerate}
    \item \textbf{Desglose}: Identifica las principales etapas, componentes o aspectos del proyecto necesarios para alcanzar el objetivo general.
    \item \textbf{Precisión}: Deben ser específicos y detallados, describiendo claramente qué se va a hacer.
    \item \textbf{Medibilidad}: Siempre que sea posible, deben ser medibles para evaluar el progreso y el éxito.
    \item \textbf{Redacción}: También se formulan utilizando verbos en infinitivo que indiquen acciones concretas, como "diseñar", "implementar", "evaluar", etc.
\end{enumerate}
\textbf{Ejemplo}:

\begin{enumerate}
    \item "Diseñar la interfaz de usuario de la plataforma para asegurar una experiencia intuitiva y amigable para los usuarios."
    \item "Implementar funciones de comunicación en tiempo real para facilitar la colaboración entre miembros del equipo."
    \item "Desarrollar herramientas de reporting avanzado para proporcionar seguimiento detallado del progreso de los proyectos."
    \item "Realizar pruebas de usabilidad para identificar y corregir posibles problemas antes del lanzamiento."
    \item "Implementar medidas de seguridad en la plataforma para proteger los datos de los usuarios y la integridad del sistema."
\end{enumerate}

\subsubsection{Pasos para Redactar los Objetivos}

\begin{enumerate}
    \item \textbf{Identifica el propósito del proyecto}: Reflexiona sobre el problema que deseas resolver o la mejora que quieres lograr.
    \item \textbf{Define el objetivo general}: En una oración clara, describe el resultado principal que esperas alcanzar.
    \item \textbf{Desglosa en objetivos específicos}: Divide el objetivo general en varias metas más pequeñas y detalladas, asegurando que cada una sea una parte integral del logro del objetivo general.
    \item \textbf{Verifica coherencia y factibilidad}: Asegúrate de que todos los objetivos específicos sean coherentes entre sí y con el objetivo general, y que sean alcanzables con los recursos y el tiempo disponibles.
    \item \textbf{Redacción final}: Redacta cada objetivo de manera clara y precisa, utilizando verbos en infinitivo para indicar acción.
\end{enumerate}

\subsubsection{Ejemplo Completo:}

\textbf{Objetivo General}: "Desarrollar una plataforma de gestión de proyectos basada en la nube para mejorar la colaboración y eficiencia en equipos distribuidos."

\textbf{Objetivos Específicos}:

\begin{enumerate}
    \item "Diseñar la interfaz de usuario de la plataforma para asegurar una experiencia intuitiva y amigable para los usuarios."
    \item "Implementar funciones de comunicación en tiempo real para facilitar la colaboración entre miembros del equipo."
    \item "Desarrollar herramientas de reporting avanzado para proporcionar seguimiento detallado del progreso de los proyectos."
    \item "Realizar pruebas de usabilidad para identificar y corregir posibles problemas antes del lanzamiento."
    \item "Implementar medidas de seguridad en la plataforma para proteger los datos de los usuarios y la integridad del sistema."
\end{enumerate}
Este enfoque asegura que el proyecto esté bien orientado y que cada fase y tarea esté claramente definida, lo que facilita su gestión y ejecución exitosa.

 

