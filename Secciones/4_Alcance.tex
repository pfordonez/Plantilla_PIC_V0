\chapter{Alcance}
\label{Alcance}

\begin{enumerate}
    \item Establecer las fases del proyecto según los objetivos específicos planteados
    \item Por cada fase establecida, se debe determinar las actividades y/o tareas en el dominio del problema.
    \item Por cada actividad y/o tarea se debe especificar las herramientas, tecnologías, métodos etc., que se pretenda usar en el dominio del proyecto.
\end{enumerate}

Ejemplo:

\begin{enumerate}
    \item \textbf{FASE}: Especificar el proceso actual de denuncias de acoso que lleva a cabo la Unidad de Bienestar Universitaria mediante el modelado BPMN
    \begin{enumerate}
        \item Estadística del uso del proceso manual y actual
        \begin{enumerate}
            \item Observación del uso manual del proceso
            \item Planificación del levantamiento de las métricas: tiempo, roles, flujos, actores usando gantt
            \item Ejecutar la planificación en un entorno web mediante googleform
            \item Analizar los resultados estadísticos 
        \end{enumerate}
        \item Construcción del modelo del proceso mediante BPM:RAD
        \begin{enumerate}
            \item Diseño de los modelos lógicos 
            \item Diseño preliminar del proceso actual 
            \item Diseño del modelo final BPM 
        \end{enumerate}
        
        
    \end{enumerate}
    
    \item \textbf{FASE}: Desarrollar el software para la web del proceso de denuncias de acoso de la Universidad Nacional de Loja mediante la metodología XP
    \begin{enumerate}
        \item Planificación de los entregables según los requisitos del proceso mediante historias de usuarios
        \item Diseño del proceso de software mediante diagrama de clases, y entidad-relación, utilizando UML y Postgersql
        \item Desarrollo los algoritmos utilizando python en el framework Django.
        \item Pruebas funcionales, de carga, de estrés y de rendimiento
    \end{enumerate}
    
    \item \textbf{FASE}: Comparar los procesos de denuncia tradicional y web mediante un ambiente de pruebas simulado para determinar los beneficios de la actualización.
     \begin{enumerate}
        \item Establecimiento del caso de prueba
        \item Planificación de la simulación manual VS web
        \item Ejecución de la simulación
        \item Análisis cuantitativo de los resultados
    \end{enumerate}
    
\end{enumerate}

