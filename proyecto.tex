% Plantilla para los PICs 
% Version 0.1
% 02/06/2022
% Por: pfordonez@unl.edu.ec 


% Elige si deseas optimizar la ejecución del proyecto almacenando las figuras generadas con TikZ y PGF en una carpeta (archivos/figuras-procesadas).
% 1 - Si, 2 - No
\def\OptimizaTikZ{2}
\input{Config/config_book.tex}
\usepackage{soul}
\usepackage{pdfpages}
\usepackage[subpreambles=false]{standalone}
\usepackage{import}
\usepackage{pgfgantt}
\usepackage{longtable}
\usepackage{graphicx}
\usepackage[hidelinks]{hyperref}

%Codigo asignado del proyecto
\newcommand{\codPTT}{}
%Título y subtítulo
\newcommand{\titulo}{Instructivo para proyectos de investigación de integración curricular}

\newcommand{\tituloEng}{Guidelines for degree project work}


\newcommand{\subtitulo}{Itinerario: Ingeniería de Software/IA/CA}
% Datos del autor
\newcommand{\miNombre}{Nombre del autor}
\newcommand{\miEmail}{autor@unl.edu.ec}
% Datos del tutor/es
\newcommand{\miTutor}{Pablo F. Ordoñez-Ordoñez, Mg.Sc.}
\newcommand{\miTutorB}{Tutor 2}

% Información añadida a las propiedades del archivo PDF.
\hypersetup{
pdfauthor = {\miNombre~(\miEmail)},
pdftitle = {\titulo},
}

%%
% Archivo de acrónimos
%%
\makeglossaries % Genera la base de datos de acrónimos
\input{Anexos/acronimos.tex} % Archivo que contiene los acrónimos


\newacronym{tt}{TT}{Trabajo de Titulación}
\begin{document}

%portada 
\import{Portada/}{portadaUNL}
% Números romanos hasta el mainmatter.
\frontmatter
\noindent {\fontsize{16pt}{16pt} \selectfont \textbf{Certificación de Tutoría}}\\

En calidad de Tutor y Cotutor del Proyecto de Trabajo de Titulación PTT, certificamos la tutela a \miNombre, con el tema \textbf{\titulo} - \textbf{\tituloEng}, quien ha cumplido con todas las observaciones requeridas. Es todo cuanto puedo decir en honor a la verdad, facultando al interesado hacer uso de la presente, así como el trámite de pertinencia del presente proyecto.\\
\linebreak 
\linebreak 
\linebreak
\rightline{Loja, \today}
\linebreak
\linebreak
\linebreak
\linebreak
\linebreak
\linebreak
\linebreak
\begin{center}
Atentamente,\\
\miTutor\\
\textbf{TUTOR}
\linebreak 
\linebreak 
\linebreak 
\linebreak 
\linebreak 
\miTutorB\\
\textbf{COTUTOR}
\end{center}


\newpage
\noindent {\fontsize{16pt}{16pt} \selectfont \textbf{Certificación de Autoría del Proyecto}}\\

Yo \miNombre, estudiante de la Universidad Nacional de Loja, declaro en forma libre y voluntaria que el presente Proyecto de Trabajo de Titulación que versa sobre \textbf{\titulo}- \textbf{\tituloEng}, así como la expresiones vertidas en la misma son autoría del compareciente, quien ha realizado en base a recopilación bibliográfica primaria y secundaria. En consecuencia asumo la responsabilidad de la originalidad de la misma y el cuidado al remitirse a las fuentes bibliográficas respectivas para fundamentar el contenido expuesto.\\
\linebreak 
\linebreak 
\linebreak 
\linebreak 
\linebreak 
\linebreak 
Atentamente,\\
\miNombre

\newpage
\tableofcontents
\newpage
\listoffigures
\newpage
\listoftables

\newpage
\vspace*{6cm}
\begin{center}
\addtolength{\baselineskip}{\baselineskip}
\begin{Huge}
\textbf {\titulo}
\linebreak 
\linebreak 
\textbf{\tituloEng}
\linebreak 
\linebreak 
\linebreak 
\subtitulo
\end{Huge}
\end{center}

% Inicia la numeración habitual.
\mainmatter

\chapter{Problema de investigación}
\label{Problemática}

\section{Situación Problemática}
\begin{itemize}
    \item Primer párrafo en donde se sitúa la definición de las variable(s)
    \item Segundo párrafo que contenga un abstracto bibliométrico sobre la variable (estado del arte RSL) (Ej. \textbf{5-10 estudios seleccionados})
    \item Contexto del primer estudio (causa/efecto..)
    \item Contexto del segundo estudio (causa/efecto)
    \item Contexto ...... [Número de estudios que hayan citado]
    \item Penúltimo párrafo --> El efecto mas trascendental que impactaría el problema (variable de enfoque)
    \item Finalmente, una solución generalizada (de la variable)
\end{itemize}


\section{Pregunta de Investigación}

\textbf{La interrogante:} ¿? es la pregunta clave que se planteará.

\textbf{Variable(1) o variables:} la variable o variables que forman parte del estudio. En el caso de un estudio descriptivo será una variable, mientras que en un estudio correlacional serán dos variables.

\textbf{Enlace o relacionante:} el vínculo con el cual se relaciona las variables.

\textbf{Población:} es generalmente la colección de individuos u objetos que son el foco principal de la investigación científica, y que serán observados, encuestados o medidos.

\textbf{Delimitación espacial:} el lugar o zona geográfica que comprende el estudio. También comprende el ámbito específico de estudio, como por ejemplo puede ser una empresa determinada o conjunto de negocios (como los cinemas).

\textbf{Delimitación temporal:} el período de tiempo que comprende el estudio.


%%%%%%%%%%%%%%%%Ejemplo%%%%%%%%%%%%


%"¿Cómo afecta la implementación de una plataforma de gestión de proyectos basada en la nube la colaboración y eficiencia de equipos distribuidos en comparación con las herramientas tradicionales de gestión de proyectos?" 

    Cuales son las PI derivadas ¿ ?
        \begin{itemize}
            \item Como se mide la afectacion (Test de impacto denomidao TA) var cuantitativa 
            \item Cual es la eficiencia (X,Z)
            \item Como establecer o medir la colaboracion (X, Y, Z)
            \item Cuales seria las herramientas tardicionales (Scrum, RUP)
        \end{itemize}


    ¿En que medida de porcetaje la el test de aceptación  afecta la implementación de una plataforma de gestión de proyectos basada en la nube la colaboración y eficiencia de equipos distribuidos en comparación con las herramientas tradicionales de gestión de proyectos como Kanban?

 \textbf{ ¿Que medida en porcentaje resulta el test de aceptación que  afecta la implementación de una plataforma de gestión de proyectos basada en la nube de colaboración Kanbam?}


\textbf{Medición objetiva}: El porcentaje del test de aceptación para la afeccion de la implementación.
    \item \textbf{Evaluación de impacto}: La aceptación del equipo de la herramienta Kanbam
    \item \textbf{Relevancia práctica}:  Nuevos resultados en las tareas  alineado con los objetivos del proyecto.


 
\chapter{Justificación}
\label{Justificación}

%Proyecto: Desarrollo de una Plataforma de Gestión de Proyectos Basada en la Nube}

\paragraph{Justificación Teórica}
La justificación teórica se refiere a la base conceptual y los principios científicos que sustentan el proyecto. Incluye: 

\begin{itemize}
    \item \textbf{Revisión de la literatura}: Se revisan estudios y publicaciones sobre gestión de proyectos, incluyendo metodologías ágiles como  Kanban, así como teorías sobre la colaboración en equipos distribuidos.
    \item \textbf{Fundamentos teóricos}: Se fundamenta en teorías de gestión de proyectos, colaboración en línea, y computación en la nube. Se citan trabajos de autores como Harold Kerzner en gestión de proyectos y estudios recientes sobre la eficacia de herramientas de colaboración en la nube.
    \item \textbf{Estado del arte}: Análisis de plataformas actuales de gestión de proyectos como Jira, Asana y Trello, identificando sus fortalezas y debilidades.
\end{itemize}
\textbf{Ejemplo de contenido}: "Las teorías de gestión de proyectos de Harold Kerzner (2017) subrayan la importancia de la planificación y la colaboración efectiva. Estudios recientes (Smith et al., 2020) muestran que las herramientas de gestión basadas en la nube pueden aumentar la eficiencia en equipos distribuidos en un 25\%."

\paragraph{Justificación Práctica}
La justificación práctica se enfoca en la utilidad y la aplicabilidad del proyecto. Se trata de demostrar la relevancia y la necesidad del proyecto en un contexto real. Incluye: 

\begin{itemize}
    \item \textbf{Problemas o necesidades}: Identificación de la necesidad de una plataforma que permita a los equipos distribuidos gestionar proyectos de manera más efectiva, con características específicas como la integración de comunicación en tiempo real y herramientas avanzadas de reporting.
    \item \textbf{Beneficios}: La plataforma propuesta permitirá una mayor eficiencia en la gestión de tareas, mejor comunicación entre los miembros del equipo y un seguimiento más preciso del progreso del proyecto. Esto reducirá los costos operativos y mejorará la productividad.
    \item \textbf{Aplicaciones}: Utilización en empresas de tecnología, agencias de marketing, y cualquier organización que trabaje con equipos distribuidos o gestione múltiples proyectos simultáneamente.
\end{itemize}
\textbf{Ejemplo de contenido}: "En una encuesta realizada por Project Management Institute (2019), el 60\% de las empresas informaron retrasos significativos en proyectos debido a la falta de herramientas de gestión adecuadas. Nuestra plataforma abordará esta necesidad proporcionando una solución integral que combina gestión de tareas, comunicación en tiempo real y reporting avanzado."

\paragraph{Justificación Metodológica}
La justificación metodológica describes los métodos y técnicas que se utilizaran para llevar cabo el proyecto. Incluye: 

\begin{itemize}
    \item \textbf{Métodos de investigación}: Uso de encuestas y entrevistas con potenciales usuarios para identificar características clave y requerimientos. Análisis de datos para entender las necesidades y preferencias de los usuarios.
    \item \textbf{Técnicas y herramientas}: Desarrollo ágil utilizando Scrum. Herramientas de desarrollo como React para el frontend, Node.js para el backend y servicios de AWS para la infraestructura en la nube.
    \item \textbf{Plan de trabajo}: El proyecto se dividirá en varias fases: investigación inicial (1 mes), diseño y planificación (2 meses), desarrollo (6 meses), pruebas (2 meses) y despliegue (1 mes). Cada fase incluirá iteraciones y revisiones periódicas con los stakeholders.
\end{itemize}
\textbf{Ejemplo de contenido}: "Utilizaremos la metodología Scrum para asegurar un desarrollo iterativo e incremental, con sprints de 2 semanas y revisiones periódicas. Las herramientas elegidas, como React y Node.js, permitirán una construcción rápida y flexible del frontend y backend, mientras que AWS proporcionará una infraestructura escalable y segura."

Este enfoque integral asegura que el proyecto esté bien fundamentado teóricamente, responda a necesidades prácticas reales y esté metodológicamente preparado para su desarrollo y éxito.

 

\chapter{Objetivos}
\label{Objetivos}

\subsubsection{Objetivo General}

\paragraph{¿Qué es?}

El objetivo general describe de manera amplia y comprensiva lo que se pretende alcanzar con el proyecto. Representa el propósito principal y la meta final del proyecto.

\paragraph{¿Cómo se elabora?}

\begin{enumerate}
    \item \textbf{Claridad}: Debe ser claro y comprensible. Evita el uso de términos ambiguos.
    \item \textbf{Amplitud}: Debe abarcar el alcance completo del proyecto.
    \item \textbf{Realismo}: Debe ser alcanzable y realista dentro del contexto y los recursos disponibles.
    \item \textbf{Redacción}: Generalmente se formula utilizando un verbo en infinitivo que indique una acción amplia, como "desarrollar", "mejorar", "crear", etc.
\end{enumerate}
\textbf{Ejemplo}: "Desarrollar una plataforma de gestión de proyectos basada en la nube para mejorar la colaboración y eficiencia en equipos distribuidos."

\subsubsection{Objetivos Específicos}

\paragraph{¿Qué son?}

Los objetivos específicos desglosan el objetivo general en metas más concretas y detalladas. Cada objetivo específico aborda un aspecto particular del proyecto y contribuye al logro del objetivo general.

\paragraph{¿Cómo se elaboran?}

\begin{enumerate}
    \item \textbf{Desglose}: Identifica las principales etapas, componentes o aspectos del proyecto necesarios para alcanzar el objetivo general.
    \item \textbf{Precisión}: Deben ser específicos y detallados, describiendo claramente qué se va a hacer.
    \item \textbf{Medibilidad}: Siempre que sea posible, deben ser medibles para evaluar el progreso y el éxito.
    \item \textbf{Redacción}: También se formulan utilizando verbos en infinitivo que indiquen acciones concretas, como "diseñar", "implementar", "evaluar", etc.
\end{enumerate}
\textbf{Ejemplo}:

\begin{enumerate}
    \item "Diseñar la interfaz de usuario de la plataforma para asegurar una experiencia intuitiva y amigable para los usuarios."
    \item "Implementar funciones de comunicación en tiempo real para facilitar la colaboración entre miembros del equipo."
    \item "Desarrollar herramientas de reporting avanzado para proporcionar seguimiento detallado del progreso de los proyectos."
    \item "Realizar pruebas de usabilidad para identificar y corregir posibles problemas antes del lanzamiento."
    \item "Implementar medidas de seguridad en la plataforma para proteger los datos de los usuarios y la integridad del sistema."
\end{enumerate}

\subsubsection{Pasos para Redactar los Objetivos}

\begin{enumerate}
    \item \textbf{Identifica el propósito del proyecto}: Reflexiona sobre el problema que deseas resolver o la mejora que quieres lograr.
    \item \textbf{Define el objetivo general}: En una oración clara, describe el resultado principal que esperas alcanzar.
    \item \textbf{Desglosa en objetivos específicos}: Divide el objetivo general en varias metas más pequeñas y detalladas, asegurando que cada una sea una parte integral del logro del objetivo general.
    \item \textbf{Verifica coherencia y factibilidad}: Asegúrate de que todos los objetivos específicos sean coherentes entre sí y con el objetivo general, y que sean alcanzables con los recursos y el tiempo disponibles.
    \item \textbf{Redacción final}: Redacta cada objetivo de manera clara y precisa, utilizando verbos en infinitivo para indicar acción.
\end{enumerate}

\subsubsection{Ejemplo Completo:}

\textbf{Objetivo General}: "Desarrollar una plataforma de gestión de proyectos basada en la nube para mejorar la colaboración y eficiencia en equipos distribuidos."

\textbf{Objetivos Específicos}:

\begin{enumerate}
    \item "Diseñar la interfaz de usuario de la plataforma para asegurar una experiencia intuitiva y amigable para los usuarios."
    \item "Implementar funciones de comunicación en tiempo real para facilitar la colaboración entre miembros del equipo."
    \item "Desarrollar herramientas de reporting avanzado para proporcionar seguimiento detallado del progreso de los proyectos."
    \item "Realizar pruebas de usabilidad para identificar y corregir posibles problemas antes del lanzamiento."
    \item "Implementar medidas de seguridad en la plataforma para proteger los datos de los usuarios y la integridad del sistema."
\end{enumerate}
Este enfoque asegura que el proyecto esté bien orientado y que cada fase y tarea esté claramente definida, lo que facilita su gestión y ejecución exitosa.

 


%\chapter{Alcance}
\label{Alcance}

\begin{enumerate}
    \item Establecer las fases del proyecto según los objetivos específicos planteados
    \item Por cada fase establecida, se debe determinar las actividades y/o tareas en el dominio del problema.
    \item Por cada actividad y/o tarea se debe especificar las herramientas, tecnologías, métodos etc., que se pretenda usar en el dominio del proyecto.
\end{enumerate}

Ejemplo:

\begin{enumerate}
    \item \textbf{FASE}: Especificar el proceso actual de denuncias de acoso que lleva a cabo la Unidad de Bienestar Universitaria mediante el modelado BPMN
    \begin{enumerate}
        \item Estadística del uso del proceso manual y actual
        \begin{enumerate}
            \item Observación del uso manual del proceso
            \item Planificación del levantamiento de las métricas: tiempo, roles, flujos, actores usando gantt
            \item Ejecutar la planificación en un entorno web mediante googleform
            \item Analizar los resultados estadísticos 
        \end{enumerate}
        \item Construcción del modelo del proceso mediante BPM:RAD
        \begin{enumerate}
            \item Diseño de los modelos lógicos 
            \item Diseño preliminar del proceso actual 
            \item Diseño del modelo final BPM 
        \end{enumerate}
        
        
    \end{enumerate}
    
    \item \textbf{FASE}: Desarrollar el software para la web del proceso de denuncias de acoso de la Universidad Nacional de Loja mediante la metodología XP
    \begin{enumerate}
        \item Planificación de los entregables según los requisitos del proceso mediante historias de usuarios
        \item Diseño del proceso de software mediante diagrama de clases, y entidad-relación, utilizando UML y Postgersql
        \item Desarrollo los algoritmos utilizando python en el framework Django.
        \item Pruebas funcionales, de carga, de estrés y de rendimiento
    \end{enumerate}
    
    \item \textbf{FASE}: Comparar los procesos de denuncia tradicional y web mediante un ambiente de pruebas simulado para determinar los beneficios de la actualización.
     \begin{enumerate}
        \item Establecimiento del caso de prueba
        \item Planificación de la simulación manual VS web
        \item Ejecución de la simulación
        \item Análisis cuantitativo de los resultados
    \end{enumerate}
    
\end{enumerate}


\chapter{Marco Teórico}
\label{MarcoTeorico}
\chapter{Metodología}
% Please add the following required packages to your document preamble:
% \usepackage{graphicx}

\chapter{Cronograma}
\label{Cronograma}
Para mas información revisar \cite{gantt}
\import{Secciones/}{7_Cronograma}
\chapter{Presupuesto y Financiamiento}

\subsection{Talento Humano}
(Rol, Personas, Tiempo (horas), Precio/Hora, Valor Total)
    \begin{itemize}
        \item Autores
        \item Director
        \item Tutores
        \item Expertos, consultores
    \end{itemize}

\subsection{Servicios}
(Servicio, Tiempo (meses), valor unitario, Valor Total)

\begin{itemize}
    \item Servicios básicos
    \item otros (especificar)
\end{itemize}

\subsection{Hardware}
(Recurso Cantidad valor unitario Valor Total)

\subsection{Software}
(Recurso Cantidad valor unitario Valor Total)

\subsection{Materiales de oficina}
(Material, Cantidad, valor unitario, Valor Total)


\subsection{Resumen del Presupuesto}


\label{Presupuesto}
 
\bibliographystyle{ieeetr}%Used BibTeX style
\bibliography{Config/bib}

%%%%
% CONTENIDO. LISTA DE ACRÓNIMOS. Comenta las líneas si no lo deseas incluir.
%%%%
% Incluye el listado de acrónimos utilizados en el trabajo. 
\printglossary[style=modsuper,type=\acronymtype,title={Lista de Acrónimos y Abreviaturas}]
% Añade el resto de acrónimos si así se desea. Si no elimina el comando siguiente
\glsaddallunused 

%%%%
% CONTENIDO. Anexos - Añade o elimina según tus necesidades
%%%%
\appendix % Inicio de los apéndices
\input{Anexos/anexo_2}
\input{Anexos/anexo_3}
\input{Anexos/anexo_I}
\end{document}